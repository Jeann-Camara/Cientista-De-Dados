% Options for packages loaded elsewhere
\PassOptionsToPackage{unicode}{hyperref}
\PassOptionsToPackage{hyphens}{url}
%
\documentclass[
]{article}
\usepackage{lmodern}
\usepackage{amssymb,amsmath}
\usepackage{ifxetex,ifluatex}
\ifnum 0\ifxetex 1\fi\ifluatex 1\fi=0 % if pdftex
  \usepackage[T1]{fontenc}
  \usepackage[utf8]{inputenc}
  \usepackage{textcomp} % provide euro and other symbols
\else % if luatex or xetex
  \usepackage{unicode-math}
  \defaultfontfeatures{Scale=MatchLowercase}
  \defaultfontfeatures[\rmfamily]{Ligatures=TeX,Scale=1}
\fi
% Use upquote if available, for straight quotes in verbatim environments
\IfFileExists{upquote.sty}{\usepackage{upquote}}{}
\IfFileExists{microtype.sty}{% use microtype if available
  \usepackage[]{microtype}
  \UseMicrotypeSet[protrusion]{basicmath} % disable protrusion for tt fonts
}{}
\makeatletter
\@ifundefined{KOMAClassName}{% if non-KOMA class
  \IfFileExists{parskip.sty}{%
    \usepackage{parskip}
  }{% else
    \setlength{\parindent}{0pt}
    \setlength{\parskip}{6pt plus 2pt minus 1pt}}
}{% if KOMA class
  \KOMAoptions{parskip=half}}
\makeatother
\usepackage{xcolor}
\IfFileExists{xurl.sty}{\usepackage{xurl}}{} % add URL line breaks if available
\IfFileExists{bookmark.sty}{\usepackage{bookmark}}{\usepackage{hyperref}}
\hypersetup{
  hidelinks,
  pdfcreator={LaTeX via pandoc}}
\urlstyle{same} % disable monospaced font for URLs
\usepackage[margin=1in]{geometry}
\usepackage{color}
\usepackage{fancyvrb}
\newcommand{\VerbBar}{|}
\newcommand{\VERB}{\Verb[commandchars=\\\{\}]}
\DefineVerbatimEnvironment{Highlighting}{Verbatim}{commandchars=\\\{\}}
% Add ',fontsize=\small' for more characters per line
\usepackage{framed}
\definecolor{shadecolor}{RGB}{248,248,248}
\newenvironment{Shaded}{\begin{snugshade}}{\end{snugshade}}
\newcommand{\AlertTok}[1]{\textcolor[rgb]{0.94,0.16,0.16}{#1}}
\newcommand{\AnnotationTok}[1]{\textcolor[rgb]{0.56,0.35,0.01}{\textbf{\textit{#1}}}}
\newcommand{\AttributeTok}[1]{\textcolor[rgb]{0.77,0.63,0.00}{#1}}
\newcommand{\BaseNTok}[1]{\textcolor[rgb]{0.00,0.00,0.81}{#1}}
\newcommand{\BuiltInTok}[1]{#1}
\newcommand{\CharTok}[1]{\textcolor[rgb]{0.31,0.60,0.02}{#1}}
\newcommand{\CommentTok}[1]{\textcolor[rgb]{0.56,0.35,0.01}{\textit{#1}}}
\newcommand{\CommentVarTok}[1]{\textcolor[rgb]{0.56,0.35,0.01}{\textbf{\textit{#1}}}}
\newcommand{\ConstantTok}[1]{\textcolor[rgb]{0.00,0.00,0.00}{#1}}
\newcommand{\ControlFlowTok}[1]{\textcolor[rgb]{0.13,0.29,0.53}{\textbf{#1}}}
\newcommand{\DataTypeTok}[1]{\textcolor[rgb]{0.13,0.29,0.53}{#1}}
\newcommand{\DecValTok}[1]{\textcolor[rgb]{0.00,0.00,0.81}{#1}}
\newcommand{\DocumentationTok}[1]{\textcolor[rgb]{0.56,0.35,0.01}{\textbf{\textit{#1}}}}
\newcommand{\ErrorTok}[1]{\textcolor[rgb]{0.64,0.00,0.00}{\textbf{#1}}}
\newcommand{\ExtensionTok}[1]{#1}
\newcommand{\FloatTok}[1]{\textcolor[rgb]{0.00,0.00,0.81}{#1}}
\newcommand{\FunctionTok}[1]{\textcolor[rgb]{0.00,0.00,0.00}{#1}}
\newcommand{\ImportTok}[1]{#1}
\newcommand{\InformationTok}[1]{\textcolor[rgb]{0.56,0.35,0.01}{\textbf{\textit{#1}}}}
\newcommand{\KeywordTok}[1]{\textcolor[rgb]{0.13,0.29,0.53}{\textbf{#1}}}
\newcommand{\NormalTok}[1]{#1}
\newcommand{\OperatorTok}[1]{\textcolor[rgb]{0.81,0.36,0.00}{\textbf{#1}}}
\newcommand{\OtherTok}[1]{\textcolor[rgb]{0.56,0.35,0.01}{#1}}
\newcommand{\PreprocessorTok}[1]{\textcolor[rgb]{0.56,0.35,0.01}{\textit{#1}}}
\newcommand{\RegionMarkerTok}[1]{#1}
\newcommand{\SpecialCharTok}[1]{\textcolor[rgb]{0.00,0.00,0.00}{#1}}
\newcommand{\SpecialStringTok}[1]{\textcolor[rgb]{0.31,0.60,0.02}{#1}}
\newcommand{\StringTok}[1]{\textcolor[rgb]{0.31,0.60,0.02}{#1}}
\newcommand{\VariableTok}[1]{\textcolor[rgb]{0.00,0.00,0.00}{#1}}
\newcommand{\VerbatimStringTok}[1]{\textcolor[rgb]{0.31,0.60,0.02}{#1}}
\newcommand{\WarningTok}[1]{\textcolor[rgb]{0.56,0.35,0.01}{\textbf{\textit{#1}}}}
\usepackage{graphicx,grffile}
\makeatletter
\def\maxwidth{\ifdim\Gin@nat@width>\linewidth\linewidth\else\Gin@nat@width\fi}
\def\maxheight{\ifdim\Gin@nat@height>\textheight\textheight\else\Gin@nat@height\fi}
\makeatother
% Scale images if necessary, so that they will not overflow the page
% margins by default, and it is still possible to overwrite the defaults
% using explicit options in \includegraphics[width, height, ...]{}
\setkeys{Gin}{width=\maxwidth,height=\maxheight,keepaspectratio}
% Set default figure placement to htbp
\makeatletter
\def\fps@figure{htbp}
\makeatother
\setlength{\emergencystretch}{3em} % prevent overfull lines
\providecommand{\tightlist}{%
  \setlength{\itemsep}{0pt}\setlength{\parskip}{0pt}}
\setcounter{secnumdepth}{-\maxdimen} % remove section numbering

\author{}
\date{\vspace{-2.5em}}

\begin{document}

\hypertarget{projeto-2---avaliauxe7uxe3o-de-risco-de-cruxe9dito}{%
\subsection{Projeto 2 - Avaliação de Risco de
Crédito}\label{projeto-2---avaliauxe7uxe3o-de-risco-de-cruxe9dito}}

Para esta análise, vamos usar um conjunto de dados German Credit Data,
já devidamente limpo e organizado para a criação do modelo preditivo.

Todo o projeto será descrito de acordo com suas etapas.

\hypertarget{etapa-1---coletando-os-dados}{%
\subsection{Etapa 1 - Coletando os
Dados}\label{etapa-1---coletando-os-dados}}

Aqui está a coleta de dados, neste caso um arquivo csv.

\begin{Shaded}
\begin{Highlighting}[]
\CommentTok{# Coletando dados}
\NormalTok{credit.df <-}\StringTok{ }\KeywordTok{read.csv}\NormalTok{(}\StringTok{"credit_dataset.csv"}\NormalTok{, }\DataTypeTok{header =} \OtherTok{TRUE}\NormalTok{, }\DataTypeTok{sep =} \StringTok{","}\NormalTok{)}
\end{Highlighting}
\end{Shaded}

\hypertarget{etapa-2---normalizando-os-dados}{%
\subsection{Etapa 2 - Normalizando os
Dados}\label{etapa-2---normalizando-os-dados}}

\begin{Shaded}
\begin{Highlighting}[]
\CommentTok{## Convertendo as variáveis para o tipo fator (categórica)}
\NormalTok{to.factors <-}\StringTok{ }\ControlFlowTok{function}\NormalTok{(df, variables)\{}
  \ControlFlowTok{for}\NormalTok{ (variable }\ControlFlowTok{in}\NormalTok{ variables)\{}
\NormalTok{    df[[variable]] <-}\StringTok{ }\KeywordTok{as.factor}\NormalTok{(df[[variable]])}
\NormalTok{  \}}
  \KeywordTok{return}\NormalTok{(df)}
\NormalTok{\}}

\CommentTok{## Normalização}
\NormalTok{scale.features <-}\StringTok{ }\ControlFlowTok{function}\NormalTok{(df, variables)\{}
  \ControlFlowTok{for}\NormalTok{ (variable }\ControlFlowTok{in}\NormalTok{ variables)\{}
\NormalTok{    df[[variable]] <-}\StringTok{ }\KeywordTok{scale}\NormalTok{(df[[variable]], }\DataTypeTok{center=}\NormalTok{T, }\DataTypeTok{scale=}\NormalTok{T)}
\NormalTok{  \}}
  \KeywordTok{return}\NormalTok{(df)}
\NormalTok{\}}

\CommentTok{# Normalizando as variáveis}
\NormalTok{numeric.vars <-}\StringTok{ }\KeywordTok{c}\NormalTok{(}\StringTok{"credit.duration.months"}\NormalTok{, }\StringTok{"age"}\NormalTok{, }\StringTok{"credit.amount"}\NormalTok{)}
\NormalTok{credit.df <-}\StringTok{ }\KeywordTok{scale.features}\NormalTok{(credit.df, numeric.vars)}

\CommentTok{# Variáveis do tipo fator}
\NormalTok{categorical.vars <-}\StringTok{ }\KeywordTok{c}\NormalTok{(}\StringTok{'credit.rating'}\NormalTok{, }\StringTok{'account.balance'}\NormalTok{, }\StringTok{'previous.credit.payment.status'}\NormalTok{,}
                      \StringTok{'credit.purpose'}\NormalTok{, }\StringTok{'savings'}\NormalTok{, }\StringTok{'employment.duration'}\NormalTok{, }\StringTok{'installment.rate'}\NormalTok{,}
                      \StringTok{'marital.status'}\NormalTok{, }\StringTok{'guarantor'}\NormalTok{, }\StringTok{'residence.duration'}\NormalTok{, }\StringTok{'current.assets'}\NormalTok{,}
                      \StringTok{'other.credits'}\NormalTok{, }\StringTok{'apartment.type'}\NormalTok{, }\StringTok{'bank.credits'}\NormalTok{, }\StringTok{'occupation'}\NormalTok{, }
                      \StringTok{'dependents'}\NormalTok{, }\StringTok{'telephone'}\NormalTok{, }\StringTok{'foreign.worker'}\NormalTok{)}

\NormalTok{credit.df <-}\StringTok{ }\KeywordTok{to.factors}\NormalTok{(}\DataTypeTok{df =}\NormalTok{ credit.df, }\DataTypeTok{variables =}\NormalTok{ categorical.vars)}
\end{Highlighting}
\end{Shaded}

\hypertarget{etapa-3---dividindo-os-dados-em-dados-de-treino-e-de-teste}{%
\subsection{Etapa 3 - Dividindo os dados em dados de treino e de
teste}\label{etapa-3---dividindo-os-dados-em-dados-de-treino-e-de-teste}}

\begin{Shaded}
\begin{Highlighting}[]
\CommentTok{# Dividindo os dados em treino e teste - 60:40 ratio}
\NormalTok{indexes <-}\StringTok{ }\KeywordTok{sample}\NormalTok{(}\DecValTok{1}\OperatorTok{:}\KeywordTok{nrow}\NormalTok{(credit.df), }\DataTypeTok{size =} \FloatTok{0.6} \OperatorTok{*}\StringTok{ }\KeywordTok{nrow}\NormalTok{(credit.df))}
\NormalTok{train.data <-}\StringTok{ }\NormalTok{credit.df[indexes,]}
\NormalTok{test.data <-}\StringTok{ }\NormalTok{credit.df[}\OperatorTok{-}\NormalTok{indexes,]}
\end{Highlighting}
\end{Shaded}

\hypertarget{etapa-4---feature-selection}{%
\subsection{Etapa 4 - Feature
Selection}\label{etapa-4---feature-selection}}

\begin{Shaded}
\begin{Highlighting}[]
\KeywordTok{library}\NormalTok{(caret) }
\end{Highlighting}
\end{Shaded}

\begin{verbatim}
## Loading required package: lattice
\end{verbatim}

\begin{verbatim}
## Loading required package: ggplot2
\end{verbatim}

\begin{Shaded}
\begin{Highlighting}[]
\KeywordTok{library}\NormalTok{(randomForest) }
\end{Highlighting}
\end{Shaded}

\begin{verbatim}
## randomForest 4.6-14
\end{verbatim}

\begin{verbatim}
## Type rfNews() to see new features/changes/bug fixes.
\end{verbatim}

\begin{verbatim}
## 
## Attaching package: 'randomForest'
\end{verbatim}

\begin{verbatim}
## The following object is masked from 'package:ggplot2':
## 
##     margin
\end{verbatim}

\begin{Shaded}
\begin{Highlighting}[]
\CommentTok{# Função para seleção de variáveis}
\NormalTok{run.feature.selection <-}\StringTok{ }\ControlFlowTok{function}\NormalTok{(}\DataTypeTok{num.iters=}\DecValTok{20}\NormalTok{, feature.vars, class.var)\{}
  \KeywordTok{set.seed}\NormalTok{(}\DecValTok{10}\NormalTok{)}
\NormalTok{  variable.sizes <-}\StringTok{ }\DecValTok{1}\OperatorTok{:}\DecValTok{10}
\NormalTok{  control <-}\StringTok{ }\KeywordTok{rfeControl}\NormalTok{(}\DataTypeTok{functions =}\NormalTok{ rfFuncs, }\DataTypeTok{method =} \StringTok{"cv"}\NormalTok{, }
                        \DataTypeTok{verbose =} \OtherTok{FALSE}\NormalTok{, }\DataTypeTok{returnResamp =} \StringTok{"all"}\NormalTok{, }
                        \DataTypeTok{number =}\NormalTok{ num.iters)}
\NormalTok{  results.rfe <-}\StringTok{ }\KeywordTok{rfe}\NormalTok{(}\DataTypeTok{x =}\NormalTok{ feature.vars, }\DataTypeTok{y =}\NormalTok{ class.var, }
                     \DataTypeTok{sizes =}\NormalTok{ variable.sizes, }
                     \DataTypeTok{rfeControl =}\NormalTok{ control)}
  \KeywordTok{return}\NormalTok{(results.rfe)}
\NormalTok{\}}

\CommentTok{# Executando a função}
\NormalTok{rfe.results <-}\StringTok{ }\KeywordTok{run.feature.selection}\NormalTok{(}\DataTypeTok{feature.vars =}\NormalTok{ train.data[,}\OperatorTok{-}\DecValTok{1}\NormalTok{], }
                                     \DataTypeTok{class.var =}\NormalTok{ train.data[,}\DecValTok{1}\NormalTok{])}


\CommentTok{# Visualizando os resultados}
\NormalTok{rfe.results}
\end{Highlighting}
\end{Shaded}

\begin{verbatim}
## 
## Recursive feature selection
## 
## Outer resampling method: Cross-Validated (20 fold) 
## 
## Resampling performance over subset size:
## 
##  Variables Accuracy  Kappa AccuracySD KappaSD Selected
##          1   0.6957 0.2726    0.09741  0.2139         
##          2   0.7303 0.3432    0.10152  0.2372         
##          3   0.7334 0.3424    0.08539  0.2067         
##          4   0.7487 0.3804    0.07602  0.1953         
##          5   0.7546 0.3948    0.07234  0.1868         
##          6   0.7549 0.3913    0.05326  0.1392         
##          7   0.7415 0.3584    0.06744  0.1712         
##          8   0.7445 0.3645    0.05526  0.1388         
##          9   0.7362 0.3409    0.05879  0.1502         
##         10   0.7480 0.3764    0.05263  0.1376         
##         20   0.7649 0.4029    0.07008  0.1810        *
## 
## The top 5 variables (out of 20):
##    account.balance, credit.duration.months, previous.credit.payment.status, savings, credit.amount
\end{verbatim}

\begin{Shaded}
\begin{Highlighting}[]
\KeywordTok{varImp}\NormalTok{((rfe.results))}
\end{Highlighting}
\end{Shaded}

\begin{verbatim}
##                                   Overall
## account.balance                18.6268425
## credit.duration.months         10.3946319
## previous.credit.payment.status  8.6114111
## savings                         7.0245412
## credit.amount                   6.7999181
## age                             4.2338202
## current.assets                  3.5805085
## apartment.type                  2.6150079
## marital.status                  2.5846166
## foreign.worker                  2.4459828
## guarantor                       2.3921475
## bank.credits                    1.9616284
## installment.rate                1.8537556
## occupation                      1.7482250
## employment.duration             1.4869947
## dependents                      0.7620542
## residence.duration              0.7359701
## telephone                       0.4381283
## credit.purpose                  0.1251003
## other.credits                  -0.5366483
\end{verbatim}

\hypertarget{etapa-5---criando-e-avaliando-a-primeira-versuxe3o-do-modelo}{%
\subsection{Etapa 5 - Criando e Avaliando a Primeira Versão do
Modelo}\label{etapa-5---criando-e-avaliando-a-primeira-versuxe3o-do-modelo}}

\begin{Shaded}
\begin{Highlighting}[]
\CommentTok{# Criando e Avaliando o Modelo}
\KeywordTok{library}\NormalTok{(caret) }
\KeywordTok{library}\NormalTok{(ROCR) }
\end{Highlighting}
\end{Shaded}

\begin{verbatim}
## Loading required package: gplots
\end{verbatim}

\begin{verbatim}
## 
## Attaching package: 'gplots'
\end{verbatim}

\begin{verbatim}
## The following object is masked from 'package:stats':
## 
##     lowess
\end{verbatim}

\begin{Shaded}
\begin{Highlighting}[]
\CommentTok{# Biblioteca de utilitários para construção de gráficos}
\KeywordTok{source}\NormalTok{(}\StringTok{"plot_utils.R"}\NormalTok{) }

\CommentTok{## separate feature and class variables}
\NormalTok{test.feature.vars <-}\StringTok{ }\NormalTok{test.data[,}\OperatorTok{-}\DecValTok{1}\NormalTok{]}
\NormalTok{test.class.var <-}\StringTok{ }\NormalTok{test.data[,}\DecValTok{1}\NormalTok{]}

\CommentTok{# Construindo um modelo de regressão logística}
\NormalTok{formula.init <-}\StringTok{ "credit.rating ~ ."}
\NormalTok{formula.init <-}\StringTok{ }\KeywordTok{as.formula}\NormalTok{(formula.init)}
\NormalTok{lr.model <-}\StringTok{ }\KeywordTok{glm}\NormalTok{(}\DataTypeTok{formula =}\NormalTok{ formula.init, }\DataTypeTok{data =}\NormalTok{ train.data, }\DataTypeTok{family =} \StringTok{"binomial"}\NormalTok{)}

\CommentTok{# Visualizando o modelo}
\KeywordTok{summary}\NormalTok{(lr.model)}
\end{Highlighting}
\end{Shaded}

\begin{verbatim}
## 
## Call:
## glm(formula = formula.init, family = "binomial", data = train.data)
## 
## Deviance Residuals: 
##     Min       1Q   Median       3Q      Max  
## -2.5704  -0.7594   0.3964   0.7370   2.0959  
## 
## Coefficients:
##                                 Estimate Std. Error z value Pr(>|z|)    
## (Intercept)                      0.05164    0.89744   0.058 0.954110    
## account.balance2                 0.44404    0.27920   1.590 0.111742    
## account.balance3                 1.57535    0.27093   5.815 6.08e-09 ***
## credit.duration.months          -0.31804    0.13592  -2.340 0.019286 *  
## previous.credit.payment.status2  1.01661    0.38990   2.607 0.009124 ** 
## previous.credit.payment.status3  1.50919    0.41010   3.680 0.000233 ***
## credit.purpose2                 -0.78106    0.47389  -1.648 0.099312 .  
## credit.purpose3                 -0.99562    0.45106  -2.207 0.027293 *  
## credit.purpose4                 -1.28365    0.43113  -2.977 0.002907 ** 
## credit.amount                   -0.30406    0.15691  -1.938 0.052645 .  
## savings2                         0.58809    0.38376   1.532 0.125414    
## savings3                         0.98700    0.41754   2.364 0.018086 *  
## savings4                         0.86708    0.31794   2.727 0.006388 ** 
## employment.duration2             0.36385    0.30988   1.174 0.240331    
## employment.duration3             0.93471    0.38342   2.438 0.014776 *  
## employment.duration4             0.40786    0.35529   1.148 0.250981    
## installment.rate2                0.03554    0.38286   0.093 0.926036    
## installment.rate3               -0.55790    0.42563  -1.311 0.189945    
## installment.rate4               -0.63924    0.36377  -1.757 0.078868 .  
## marital.status3                  0.63429    0.25354   2.502 0.012357 *  
## marital.status4                  0.47457    0.41549   1.142 0.253369    
## guarantor2                       0.15681    0.35648   0.440 0.660021    
## residence.duration2             -0.71953    0.38208  -1.883 0.059673 .  
## residence.duration3             -0.36810    0.41444  -0.888 0.374439    
## residence.duration4             -0.58035    0.38399  -1.511 0.130694    
## current.assets2                 -0.11769    0.32045  -0.367 0.713419    
## current.assets3                  0.17784    0.29681   0.599 0.549066    
## current.assets4                 -0.47605    0.51222  -0.929 0.352693    
## age                              0.23348    0.13427   1.739 0.082047 .  
## other.credits2                   0.20441    0.27876   0.733 0.463371    
## apartment.type2                  0.47311    0.31063   1.523 0.127748    
## apartment.type3                  0.42870    0.59476   0.721 0.471033    
## bank.credits2                   -0.02780    0.28966  -0.096 0.923535    
## occupation2                     -0.87130    0.70784  -1.231 0.218350    
## occupation3                     -0.67518    0.67673  -0.998 0.318420    
## occupation4                     -0.94214    0.73770  -1.277 0.201553    
## dependents2                     -0.32460    0.31501  -1.030 0.302809    
## telephone2                       0.07044    0.25528   0.276 0.782606    
## foreign.worker2                  1.83660    0.81968   2.241 0.025050 *  
## ---
## Signif. codes:  0 '***' 0.001 '**' 0.01 '*' 0.05 '.' 0.1 ' ' 1
## 
## (Dispersion parameter for binomial family taken to be 1)
## 
##     Null deviance: 753.74  on 599  degrees of freedom
## Residual deviance: 561.50  on 561  degrees of freedom
## AIC: 639.5
## 
## Number of Fisher Scoring iterations: 5
\end{verbatim}

\begin{Shaded}
\begin{Highlighting}[]
\CommentTok{# Testando o modelo nos dados de teste}
\NormalTok{lr.predictions <-}\StringTok{ }\KeywordTok{predict}\NormalTok{(lr.model, test.data, }\DataTypeTok{type=}\StringTok{"response"}\NormalTok{)}
\NormalTok{lr.predictions <-}\StringTok{ }\KeywordTok{round}\NormalTok{(lr.predictions)}

\CommentTok{# Avaliando o modelo}
\KeywordTok{confusionMatrix}\NormalTok{(}\KeywordTok{table}\NormalTok{(}\DataTypeTok{data =}\NormalTok{ lr.predictions, }\DataTypeTok{reference =}\NormalTok{ test.class.var), }\DataTypeTok{positive =} \StringTok{'1'}\NormalTok{)}
\end{Highlighting}
\end{Shaded}

\begin{verbatim}
## Confusion Matrix and Statistics
## 
##     reference
## data   0   1
##    0  56  40
##    1  51 253
##                                           
##                Accuracy : 0.7725          
##                  95% CI : (0.7282, 0.8127)
##     No Information Rate : 0.7325          
##     P-Value [Acc > NIR] : 0.03831         
##                                           
##                   Kappa : 0.3999          
##                                           
##  Mcnemar's Test P-Value : 0.29451         
##                                           
##             Sensitivity : 0.8635          
##             Specificity : 0.5234          
##          Pos Pred Value : 0.8322          
##          Neg Pred Value : 0.5833          
##              Prevalence : 0.7325          
##          Detection Rate : 0.6325          
##    Detection Prevalence : 0.7600          
##       Balanced Accuracy : 0.6934          
##                                           
##        'Positive' Class : 1               
## 
\end{verbatim}

\hypertarget{etapa-6---otimizando-o-modelo}{%
\subsection{Etapa 6 - Otimizando o
Modelo}\label{etapa-6---otimizando-o-modelo}}

\begin{Shaded}
\begin{Highlighting}[]
\CommentTok{## Feature selection}
\NormalTok{formula <-}\StringTok{ "credit.rating ~ ."}
\NormalTok{formula <-}\StringTok{ }\KeywordTok{as.formula}\NormalTok{(formula)}
\NormalTok{control <-}\StringTok{ }\KeywordTok{trainControl}\NormalTok{(}\DataTypeTok{method =} \StringTok{"repeatedcv"}\NormalTok{, }\DataTypeTok{number =} \DecValTok{10}\NormalTok{, }\DataTypeTok{repeats =} \DecValTok{2}\NormalTok{)}
\NormalTok{model <-}\StringTok{ }\KeywordTok{train}\NormalTok{(formula, }\DataTypeTok{data =}\NormalTok{ train.data, }\DataTypeTok{method =} \StringTok{"glm"}\NormalTok{, }\DataTypeTok{trControl =}\NormalTok{ control)}
\NormalTok{importance <-}\StringTok{ }\KeywordTok{varImp}\NormalTok{(model, }\DataTypeTok{scale =} \OtherTok{FALSE}\NormalTok{)}
\KeywordTok{plot}\NormalTok{(importance)}
\end{Highlighting}
\end{Shaded}

\includegraphics{Projeto_02---Análise-de-Risco-de-Credito_files/figure-latex/otimizando-1.pdf}

\begin{Shaded}
\begin{Highlighting}[]
\CommentTok{# Construindo o modelo com as variáveis selecionadas}
\NormalTok{formula.new <-}\StringTok{ "credit.rating ~ account.balance + credit.purpose + previous.credit.payment.status + savings + credit.duration.months"}
\NormalTok{formula.new <-}\StringTok{ }\KeywordTok{as.formula}\NormalTok{(formula.new)}
\NormalTok{lr.model.new <-}\StringTok{ }\KeywordTok{glm}\NormalTok{(}\DataTypeTok{formula =}\NormalTok{ formula.new, }\DataTypeTok{data =}\NormalTok{ train.data, }\DataTypeTok{family =} \StringTok{"binomial"}\NormalTok{)}

\CommentTok{# Visualizando o modelo}
\KeywordTok{summary}\NormalTok{(lr.model.new)}
\end{Highlighting}
\end{Shaded}

\begin{verbatim}
## 
## Call:
## glm(formula = formula.new, family = "binomial", data = train.data)
## 
## Deviance Residuals: 
##     Min       1Q   Median       3Q      Max  
## -2.4724  -0.8771   0.5041   0.7942   1.9136  
## 
## Coefficients:
##                                 Estimate Std. Error z value Pr(>|z|)    
## (Intercept)                     -0.56635    0.47329  -1.197  0.23146    
## account.balance2                 0.43581    0.24981   1.745  0.08106 .  
## account.balance3                 1.57061    0.25047   6.271 3.59e-10 ***
## credit.purpose2                 -0.59911    0.42231  -1.419  0.15600    
## credit.purpose3                 -0.59169    0.38966  -1.518  0.12890    
## credit.purpose4                 -0.89076    0.38585  -2.309  0.02097 *  
## previous.credit.payment.status2  0.96944    0.34130   2.840  0.00450 ** 
## previous.credit.payment.status3  1.46821    0.35701   4.113 3.91e-05 ***
## savings2                         0.46753    0.35248   1.326  0.18470    
## savings3                         0.87919    0.38625   2.276  0.02283 *  
## savings4                         0.76504    0.28907   2.647  0.00813 ** 
## credit.duration.months          -0.49062    0.09927  -4.942 7.72e-07 ***
## ---
## Signif. codes:  0 '***' 0.001 '**' 0.01 '*' 0.05 '.' 0.1 ' ' 1
## 
## (Dispersion parameter for binomial family taken to be 1)
## 
##     Null deviance: 753.74  on 599  degrees of freedom
## Residual deviance: 612.82  on 588  degrees of freedom
## AIC: 636.82
## 
## Number of Fisher Scoring iterations: 4
\end{verbatim}

\begin{Shaded}
\begin{Highlighting}[]
\CommentTok{# Testando o modelo nos dados de teste}
\NormalTok{lr.predictions.new <-}\StringTok{ }\KeywordTok{predict}\NormalTok{(lr.model.new, test.data, }\DataTypeTok{type=}\StringTok{"response"}\NormalTok{) }
\NormalTok{lr.predictions.new <-}\StringTok{ }\KeywordTok{round}\NormalTok{(lr.predictions.new)}

\CommentTok{# Avaliando o modelo}
\KeywordTok{confusionMatrix}\NormalTok{(}\KeywordTok{table}\NormalTok{(}\DataTypeTok{data=}\NormalTok{lr.predictions.new, }\DataTypeTok{reference=}\NormalTok{test.class.var), }\DataTypeTok{positive=}\StringTok{'1'}\NormalTok{)}
\end{Highlighting}
\end{Shaded}

\begin{verbatim}
## Confusion Matrix and Statistics
## 
##     reference
## data   0   1
##    0  50  31
##    1  57 262
##                                           
##                Accuracy : 0.78            
##                  95% CI : (0.7362, 0.8196)
##     No Information Rate : 0.7325          
##     P-Value [Acc > NIR] : 0.016898        
##                                           
##                   Kappa : 0.3917          
##                                           
##  Mcnemar's Test P-Value : 0.007699        
##                                           
##             Sensitivity : 0.8942          
##             Specificity : 0.4673          
##          Pos Pred Value : 0.8213          
##          Neg Pred Value : 0.6173          
##              Prevalence : 0.7325          
##          Detection Rate : 0.6550          
##    Detection Prevalence : 0.7975          
##       Balanced Accuracy : 0.6807          
##                                           
##        'Positive' Class : 1               
## 
\end{verbatim}

\hypertarget{etapa-7---curva-roc-e-avaliauxe7uxe3o-final-do-modelo}{%
\subsection{Etapa 7 - Curva ROC e Avaliação Final do
Modelo}\label{etapa-7---curva-roc-e-avaliauxe7uxe3o-final-do-modelo}}

\begin{Shaded}
\begin{Highlighting}[]
\CommentTok{# Avaliando a performance do modelo}

\CommentTok{# Criando curvas ROC}
\NormalTok{lr.model.best <-}\StringTok{ }\NormalTok{lr.model}
\NormalTok{lr.prediction.values <-}\StringTok{ }\KeywordTok{predict}\NormalTok{(lr.model.best, test.feature.vars, }\DataTypeTok{type =} \StringTok{"response"}\NormalTok{)}
\NormalTok{predictions <-}\StringTok{ }\KeywordTok{prediction}\NormalTok{(lr.prediction.values, test.class.var)}
\KeywordTok{par}\NormalTok{(}\DataTypeTok{mfrow =} \KeywordTok{c}\NormalTok{(}\DecValTok{1}\NormalTok{,}\DecValTok{2}\NormalTok{))}
\KeywordTok{plot.roc.curve}\NormalTok{(predictions, }\DataTypeTok{title.text =} \StringTok{"Curva ROC"}\NormalTok{)}
\KeywordTok{plot.pr.curve}\NormalTok{(predictions, }\DataTypeTok{title.text =} \StringTok{"Curva Precision/Recall"}\NormalTok{)}
\end{Highlighting}
\end{Shaded}

\includegraphics{Projeto_02---Análise-de-Risco-de-Credito_files/figure-latex/curva-1.pdf}

\end{document}
